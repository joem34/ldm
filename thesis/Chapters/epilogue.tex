\chapter{Discussion}
%overview of results
In this section, a discussion of the results is presented, and framed in the context of other work in the field. Firstly, a brief overview of the different models is presented. Chapter~\ref{section:estimation} on estimation explores the models in more detail. The importance and relevance of the work is then discussed. Finally, the limitations of the work are presented and areas where future work is required are detailed.

\section{Overview of results}
In the estimation process, it was found that the output from the calibrated gravity model  was unrealistic to the point of being almost unusable. The correlation of between 0.36 and 0.52 across trip purposes was very low. The calibrated impedance coefficients from the gravity model were retained to build a simple discrete choice model incorporating distance, population and employment. Essentially using the same variables as the gravity model, this first destination choice model performed significantly better across all trip purposes. Further models, incorporating interactions between the origin and destination, such as metropolitan/regional connections further improved the model results. Not all interactions were found to be significant. In particular, the barrier between English and French speaking parts of Canada had only a limited impact. 

The next iteration improved the estimation of intra-zonal trips in smaller metropolitan zones. It was discovered that to properly account for intra-zonal behavior, separate parameters were needed for rural and metropolitan origins. The coefficients for these two parameters were of opposite sign, logically indicating that intra-zonal long distance travel is more likely in larger rural zones, but unlikely in metropolitan zones that are barely large enough to support an internal long distance trip. The accuracy of the model for leisure trips was noticeably weaker than for the other two models, and it was hypothesized that better modeling of the destination attractiveness for leisure activities would improve the results.

The LBSN based models demonstrated how venue checkin data from foursquare could be incorporated into a destination choice model. As hypothesized, the foursquare data dramatically improved the estimation of leisure trips. It also performed well for visit and business trips. Even with all other variables excluded except for distance, the model (\textit{m4}) performed comparably to the basic \textit{m1} model. When the foursquare data was combined with the variables from previous iterations, a very agreeable model (\textit{m5}) was produced. In the final model \textit{m6}, a second distance term was added to improve the accuracy of the model.
After consideration, income was not included in the parameters. However, based on the literature, it should still be included in other steps in the model, based on the findings in the literature. textcite{limtanakool2006participation} found gender and imcome significant for trip generation, and it is widely recognised that income affects mode choice, especially through auto ownership \parencite{ben1974some, miller1998urban, raphael2002car}.

When the model was calibrated to the average trip length, the trip distribution suffered. On closer inspection, the model still overestimates the number of intra-zonal trips within Toronto, and underestimating the inter-zonal trips between large population centers, such as Toronto, Ottawa and Montreal. The car journey from Toronto to Ottawa takes over 4 hours, while flying takes only 55 minutes. The inclusion of feedback from the mode choice model, when available, could potentially improve the estimation of these connections.

The scenario analysis of a new ski resort demonstrated how the use of LBSN data to represent destination attractiveness modeled impacts that would otherwise not have been observable with the model. The sensitivity of each parameter in the scenario is also visible. Despite the high correlation between the foursquare variables at this spatial resolution, the varying impact of different categories is still evident, and shows that parameters may need to be considered in the context of the spatial resolution of the zone system. 

%what your findings might mean
\section{Contributions}
This thesis provides multiple contributions to the field. As a completed and calibrated domestic destination choice model, it presents a significant amount of work towards a completed long distance transport model for Ontario. The superior performance of the destination choice models over the gravity model validates the work of others in this area \parencite{Mishra13}, 
and adds further weight to the argument that such disaggregate methods, despite the additional effort required, result in better transport models \parencite{sbayti2010best, lemp2007aggregate}. \textcite{bhatta2011intrazonal} found that the inclusion of intra-zonal trips is important in model estimation, and the investigations in this paper into origin-destination interactions highlight some novel ways to adjust for these in long distance modeling.

Big data already presents both exciting opportunities and daunting challenges to transportation modeling. It is predicted that by 2020, 6.1 billion people will be owning and using a smartphone~\parencite{ericsson16}. With GPS already standard on mobile devices, data is already being collected, in real time and at an unprecedented spatial resolution, that tracks individuals as they travel and interact with their environment. People are also choosing to share more about their behavior, and foursquare is just one one example of a platform that enables this. The results of the estimation process and sensitivity analysis in this thesis show how even a limited application of such data can dramatically improve destination choice models for leisure travel. The capability to model many aspects of destination attractiveness is particularly useful for leisure travel.

While there has been a "virtual explosion of data availability" ~\textcite{nagel2001workshop},  ~\textcite{horni2012improve} note that  the collection of big data such as GPS and GSM data "is generally associated with privacy, cost and technical issues".  These challenges go against the ideal of general models that are flexible and transferable \parencite{patriksson2015traffic}. None the less, big data undoubtedly has a role to play in the future of transport modeling. \textcite{rth2015} suggests further research into probabilistic models based on big data and the blending of big data with data from travel diaries. 

Traditional mobility surveys such as the TSRC still have an important role to play in transportation modeling. They track the same individuals over time, and provide the socioeconomic characteristics of the individual, which have been repeatedly shown to be important determinants of travel behavior~\parencite{pas1984effect,hanson1982determinants}. However, they also have shortcomings; they normally rely on the recall ability of the participant, and are limited in both their spatial and temporal resolutions. The TSRC data exhibited this second shortcoming, which influenced the decisions around spatial resolution in the destination choice model. Most sources of big data collect information from the individual in real time (GPS), or rely on location tracking services to verify the checkin (foursquare) to an accuracy of meters.


A higher spatial resolution in the observed trip records would allow for a more detailed destination choice model, but would also require a more detailed consideration of another issue, the choice set selection. With only 69 destinations, as used in thesis, it is not essential to restrict the choice set for an individual. However, should the number of alternatives be very large, the choice set presented to each individual needs to be reduced for each individual. Realistically, an individual is not capable of evaluating thousands of possible alternatives when selecting a destination. And any model that assumes this would be unrealistic. In particular, \textcite{AkivaLerman85} found that the the size and selection of the choice set impacts the model performance.

The venue data for each zone essentially acts as database of the points of interest (POI) at a particular destination. POI data is available from many sources, such as Open Street Maps. However, LBSNs such as foursquare take this POI database one step further, by measuring the popularity of each POI. In the case of foursquare, as discussed in section~\ref{section:foursquare}, when summed together, checkins measure the intensity of activity at each POI. A measure of importance was clearly beneficial in the model. Not all POIs are equal. Hotels are of different sizes, some national parks are more visited than others. Of course, the importance of each POI can be measured based on attributes such as the number of hotel beds or recorded visitors per year. However, the data collection required is prohibitive, particularly for a large scale model. LBSN data provides an easily accessible metric the importance of POIs, and in turn, destination utility. 

%limitations
\section{Limitations and future work}
One of the benefits of models based on socioeconomic variables is the ability to run the model for future years and model the impacts of demographic change. This is difficult for the foursquare checkin counts. Not only is it hard to predict the how the popularity of certain venues will grow or decline in future years, but the quantity of checkins depends on uptake of the foursquare platform. Further study of the demographics of foursquare users and their distribution in relation to the population. 

Many dimensions to the data which were not explored in this thesis,  particularly the temporal aspect. Through public services such as twitter, Panel data can be collected by associating checkins over time with an individual. In future work utilizing more detailed foursquare data, checkins could be filtered for those performed by residents of Canada, or grouped by season to further improve the modeling of different trip purposes.

In study on why people use foursquare, \textcite{lindqvist2011m} found that ``participants expressed reluctance to check-in at home, work, and other places that one might expect them to be at". This suggests that there at limits to how effectively foursquare can model travel behavior. A potential alternative would be to use foursquare or a similar LBSN as a POI database, and use GPS traces to identify the intensity of activity at these locations, thereby avoiding the selective reporting behavior evident in foursquare usage.


\section{Conclusions}
In conclusion, this thesis presents an estimated, calibrated and implemented multinomial logit model for long distance destination choice in Ontario, Canada. The challenges of zonal aggregation were considered, and care taken to specify an appropriate zone system for the model. Models based primarily on population and employment were found to work well for visit and business travel, but not leisure travel. A POI database was built from millions of foursquare checkins to test the hypothesis that destination attributes based geo-tagged big data can improve the modeling of destination choice. Alternative specific parameters based on the checkin data did indeed improve the model accuracy across all trip purposes, particularly leisure travel, supporting this hypothesis. The results of the scenario analysis using the fully implemented model also reinforced the importance of properly measuring destination attractiveness for leisure travel. Further work is still needed to incorporate feedback from mode choice into the model, and to allocate destinations within Ontario to the finer resolution zone system of TAZs. The application of foursquare data showed promising results, and invites further research into utilize big data in destination choice modeling.
